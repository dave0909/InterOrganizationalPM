% To allow for extra UTF-8 characters in the text
\usepackage[utf8]{inputenc}
% To use colours
\usepackage[usenames,dvipsnames]{xcolor}
% To nicely format URLs
\usepackage{url}
% For using inparaenum, basically
\usepackage{paralist}
% For adding TODO notes
\usepackage[obeyDraft,textsize=tiny,backgroundcolor=blue!10]{todonotes}
% To 
% \usepackage[square,numbers,sectionbib]{natbib}
% To add author blocks to the front-matter
\usepackage{authblk}
% To create linked anchors on top of references
%\usepackage[pdfborder={0 0 0}]{hyperref}
% To play around with enumerations and bullet lists
\usepackage{enumitem}
% To load hyphenation rules and other Locale-standardised things
\usepackage[british]{babel}
% For the letter-like symbol, {\Letter}
\usepackage{marvosym}
% To comment
\usepackage{comment}
% To adjust margins
\usepackage{geometry}
% Fancy HREFs
\usepackage[draft=true]{hyperref}
%\usepackage{nohyperref}
% Fancy enumerations and itemised lists
\usepackage{enumitem}
% For \mathbb command, among the others
\usepackage{amsfonts}
% For creating side-notes
\usepackage{marginnote}
% For specifying kewords and acronyms
\usepackage[nonumberlist,acronym,sanitize=none]{glossaries}
\glsdisablehyper
% For equations, arrays of equations, defining operator names, etc.
\usepackage{amsmath}
% For cursive math
\usepackage{mathrsfs}
% For math symbols, such as \nexists
\usepackage{amssymb}
% To check whether the document is in its final version or not
\usepackage{ifdraft}
% To highlight text
\usepackage{soul}
% For a decent formatting of numbers (and a wonderful system for numeric columns in tables, ``S'')
\usepackage{siunitx}
% To create right arrows in the text environment
\usepackage{textcomp}
% For smart references
\usepackage[capitalise,nameinlink]{cleveref}
\crefname{algocf}{algorithm}{algorithms}
\Crefname{algocf}{Algorithm}{Algorithms}
% To cross-reference other documents
\usepackage{xr}
