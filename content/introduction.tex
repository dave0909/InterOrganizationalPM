\section{Introduction}
\label{sec:introduction}
\todo[inline]{%
	´CDC: Status corrente secondo me. Abbiamo\ldots
	Un'intro affabile ma che dobbiamo ricalibrare per toglierla dal pantano della TEE e puntare dritto al nostro obiettivo dichiarato e mantenuto, con vista sul potenziale inespresso.\\
	Un motivating scenario che però resta staccato dal resto.\\
	Un related work che però dovrebbe spostarsi sul far capire cosa facciamo di più e di diverso, o di simile ma rivisto.\\
	Una overview eccellente dell'architettura che però non è connessa con l'esempio.\\
	Una overview accurata delle interazioni dei moduli (e quando lo spazio è poco è il caso di fare il merge tra le due cose per risparmiare spazio, come ci dicemmo) -- anche qui senza esempio, dunque capire cosa si fa dove è arduo. Non per me o per te, ma perché conosciamo il lavoro. Chi non lo conosce, non ha assolutamente idea di cosa voglia dire fare il merge degli eventi da tracce provienienti da actor diversi rimettendoli in ordine così da preservare integrità temporale, per esempio.\\
	Una discussion/evaluation ben congegnata in cui però manca la parte quantitativa e la qualitativa è ancora da dettagliare.\\
	Una conclusione.%
}
In today's business landscape, organizations constantly seek ways to enhance operational efficiency, increase performance, and gain valuable insights to improve their processes. Process mining offers techniques to discover, monitor, and improve business processes by extracting knowledge from chronological records known as \textit{event logs}. Organizations record in these ledgers events referring to activities and interactions occurring within a business process. The vast majority of process mining contributions consider \textit{intra-organizational} settings, in which business processes are executed inside individual organizations. However, organizations increasingly recognize the value of collaboration and synergy in achieving operational excellence. \textit{Inter-organizational} business processes involve several independent organizations actively cooperating to achieve a shared objective. %Several ways of collaborative setting are possible: \textit{subcontracting}, \textit{chained execution}, \textit{capacity sharing}, \textit{case transfer}, \textit{loosely coupled} \cite{van1999process}.
Despite the advantages in terms of transparency, performance optimization, and benchmarking that companies can gain from such practices, inter-organizational process mining raises challenges that make it still hardly applicable. The major issue concerns confidentiality. Companies are reluctant to outsource to their partners inside information that are required to execute process mining methodologies. Indeed, the sharing of sensitive operational data across organizational boundaries introduces concerns about data privacy, security, and compliance with regulations. \textit{Trusted execution environments} (TEEs) can serve as fundamental enablers to balance the need for insights with the imperative to protect sensitive information in inter-organizational settings. TEEs offer secure contexts that guarantee code integrity and data confidentiality in foreign devices. \textit{Trusted Applications} are tamper-proof software objects running in these contexts. In this paper, we employ trusted applications running in TEEs to enable companies to execute process mining techniques and exchange sensitive information by preserving privacy and integrity of the shared data. In terms of contribution, we extend the state of the art by: (i) proposing a TEE-based infrastructure that enables process mining in inter-organizational settings; (ii) designing the core components of trusted applications providing organizations secrecy in data exchange and utilization.

The remainder of the paper is structured as follows: \cref{sec:background} provides an overview of related work inherent to the theme of inter-organizational process mining. In \cref{sec:motivating}, we introduce a use case example that considers a healthcare scenario. The high-level architecture of our solution is presented in \cref{sec:design}. Following on from this, we instantiate the addressed design principles in \cref{sec:realization} focusing on the employed technologies, workflow, and implementation. In \cref{sec:evaluation}, we discuss our solution. Finally, we conclude and present directions for future work in \cref{sec:conclusion}.
