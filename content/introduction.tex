\section{Introduction}
In today's business landscape, organizations are constantly seeking ways to enhance their operational efficiency, increase their performance, and gain valuable insights to improve their processes. Process mining offers tecniques to discover, monitor and improve business processes through the extraction of knowledge from chronological record known as \textit{event logs}. Organizations record in this ledgers events referring to activities and interactions occurring within a business process. The vast majoriy of process mining contributions considers \textit{intra-organizational} settings, in which, business processes are executed inside individual organizations. However, organizations are increasingly recognizing the value of collaboration and synergy in achieving operational excellence. \textit{Inter-organizational} business processes involve several indipendent organizations that actively cooperate to achieve a shared objective. Several ways of collaborative setting are possible: \textit{subcontracting}, \textit{chained execution}, \textit{capacity sharing}, \textit{case transfer}, \textit{loosely coupled} \cite{van1999process}. Although the advantages in terms of transparency, performance optimization and benchmarking that companies can gain from such practices, inter-organizational process mining raises challanges that make it still hardly applicable. The major issue concerns confidentiality. Companies are reluctant to outsource with their partners the inside information required to execute process mining methodologies. Hence, the sharing of sensitive operational data across organizational boundaries introduces concerns about data privacy, security, and compliance with regulations. Balancing the need for insights with the imperative to protect sensitive information adds complexity to the analysis.


%In this context, the availability of worthwhile information plays a key role. One of the primary obstacles lies in securely and reliably accessing and utilizing data from various companies or business units, ensuring that all involved parties can derive substantial benefits from it. Traditional approaches to sharing sensitive data across organizational boundaries often involve concerns related to privacy, data integrity, and the need for mutual trust. This paper introduces a novel approach that leverages Trusted Execution Environment (TEE) technology to facilitate the exchange of event logs among different companies or business units. TEE provides a secure and isolated environment within a computer system, ensuring the confidentiality, integrity, and privacy of data and code execution. By utilizing TEE, organizations can exchange event logs in a trusted and privacy-preserving manner, enabling them to harness valuable information from external processes to enhance, monitor, or modify their own processes effectively.

%The proposed methodology combines the power of TEE with process mining techniques to unlock the potential of inter-organizational event log exchange. Process mining is a data-driven approach that extracts knowledge and insights from event logs to gain a comprehensive understanding of business processes. By incorporating external event logs from trusted sources, organizations can obtain a broader and more accurate view of the end-to-end processes they are involved in, leading to better process optimization, performance monitoring, and decision-making. By bridging the gap between different organizations, the utilization of TEE-enabled event log exchange for process mining offers promising opportunities for collaborative learning, benchmarking, and continuous process improvement. The findings and insights derived from this research have the potential to revolutionize the way organizations approach process optimization and enable them to make more informed decisions based on a holistic view of their processes.


\label{sec:introduction}