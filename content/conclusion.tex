\section{Conclusion and Future Work}
\label{sec:conclusion}
Confidentiality is of paramount importance in inter-organizational process mining due to the transmission of sensitive data across organizational boundaries. Our research investigates a secrecy-preserving approach that enables organizations to employ process mining techniques with event logs from multiple organizations while ensuring the protection of privacy and confidentiality. Our solution still has room for improvement. We operate under the assumption of fair conduct by data provisioners and do not account for the presence of injected or maliciously manipulated event logs. In addition, we do not handle TEE crashes and suppose that miners and providers exchange messages in perfect communication channels where no loss, no snap, and no bit corruption occours. Additionally, our approach relies on certain assumptions about event log data, including the existence of a universal clock for event timestamps, which may not be realistic in situations where organizations are not perfectly synchronized. To address this challenge, we intend to explore a solution based on Network Time Protocols (NTP). Our future work encompasses the development of a formalized interaction protocol governing the communication between data provisioners and miners. The presented solution embraces model process mining techniques in a general way. However, we believe that the presented approach is particularly compatible with declarative model representations. Therefore, trusted applications could compute and store the entire set of rules representing a business process, and users may interact with them via trusted queries. Finally, in our implementation, we have focused on process discovery tasks. However, our approach has the potential to seamlessly cover a wider array of process mining functionalities such as \textit{conformance checking}, and \textit{performance analysis} techniques. Implementing them and show their integrability with our approach paves the path for future work. 
%\todo{CDC: The bibliography entries are too rich. Look at \cite{engel2011process}. Do we really care that the conference was in Toulouse? And look at \cite{koch2002expressive}: the acronym is enough for the conference name. Also, the volume number is useless if we do not have the series (anyway, we could not care less about either of the two). We have already gone through this, so we should shorten the entries as we know.}

\begin{comment}
Limitations:
\begin{itemize}
    \item Both producer and consumer act fairly (so we do not expect to have injected data) ok
    \item We do not manage TEE crashes
    \item We assume a perfect communication channel (no loss, no snap, no corrupted bits)
    \item Universal clock for event timestamps (cite Event log cleaning for business process analytics by Andreas Solti)
\end{itemize} 
Future Work:
\begin{itemize}
    \item Declarative models adaptation ok
    \item Output inside the TEE, interactions through trusted applications
    \item Real-world event log data ok
    \item Usage policies integration ok
    \item Formal interaction protocol ok
    \item Threat model ok
    \item Security evaluation ok
\end{itemize}
\end{comment}
%future work, quali segment portano al risultato attuale e farlo per ogni risultato intermedio -call valerio claudio 9-06-23
