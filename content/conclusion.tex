\section{Conclusion and Future Work}
\label{sec:conclusion}
Confidentiality is paramount in inter-organizational process mining, as sensitive data traverses organizational boundaries. Preserving privacy and confidentiality of operational data becomes a critical concern in this regard. Our research work explore a TEE-based solution in which trusted applications allows organizations to apply process mining tecniques using event logs from various organizations, garantendo il rispetto della riservatezza delle informazioni scambiate.

Our solution still has room for improvement in several areas. We assume that providers act fairly, and we do not expect to have injected or maliciously manipulated. In addition, we do not handle TEE crashes and assume that miners and providers exchange messages in perfect communication channels where no loss, no snap, and no bit corruption take place. We also make assumptions on event log data. Our solution presupposes a universal clock for event timestamps in different systems; hence no synchronization procedure is actually considered. Moreover, traces of different organizations referring to the same process instance are characterized by the same case identifier. This scenario is implausible in real-world data since organizations may adopt different case notations. To overcome this issue, alternative event log representations should be considered.

Future work includes the elaboration of an interaction protocol that formalizes the communication workflow between data providers and miners. Additionally, we plan to integrate usage control policies containing terms and conditions on event log utilization. To achieve this goal, we will design dedicated mechanisms inside trusted applications for monitoring usage rules and enforcing their fulfillment. The presented solution embraces model process mining techniques in a general way. However, we believe the approach presented is even more compatible with declarative model representations. Therefore, trusted applications could compute and store the entire set of rules representing a business process, and users may interact with them via trusted queries. We plan to extend the discussion of \cref{sec:evaluation} by integrating threat modeling analysis and quantitative assessments concerning scalability, throughput, and performances on real-world event logs.

\begin{comment}
Limitations:
\begin{itemize}
    \item Both producer and consumer act fairly (so we do not expect to have injected data) ok
    \item We do not manage TEE crashes
    \item We assume a perfect communication channel (no loss, no snap, no corrupted bits)
    \item Universal clock for event timestamps (cite Event log cleaning for business process analytics by Andreas Solti)
\end{itemize} 
Future Work:
\begin{itemize}
    \item Declarative models adaptation ok
    \item Output inside the TEE, interactions through trusted applications
    \item Real-world event log data ok
    \item Usage policies integration ok
    \item Formal interaction protocol ok
    \item Threat model ok
    \item Security evaluation ok
\end{itemize}


%future work, quali segment portano al risultato attuale e farlo per ogni risultato intermedio -call valerio claudio 9-06-23
\end{comment}