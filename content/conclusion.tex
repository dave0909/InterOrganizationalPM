\section{Conclusion and Future Work}
\todo[inline]{It can be reduced}
\label{sec:conclusion}

In our implementation, we have focused on process discovery tasks. However, our approach has the potential to seamlessly cover a wider array of process mining functionalities such as % supports the integration of \textit{process discovery}~\cite{citation}, 
\textit{conformance checking}, and \textit{performance analysis} techniques.
Implementing them and show their integrability with our approach paves the path for future work.
\todo{To be rephrased and repositioned wherever it fits best.}


Confidentiality is paramount in inter-organizational process mining, as sensitive data traverses organizational boundaries. Preserving the privacy and confidentiality of operational data becomes a critical concern in this regard. Our research explores a secrecy-preserving approach in which trusted applications allow organizations to apply process mining techniques using event logs from various organizations while ensuring the preservation of partners' privacy. Our solution still has room for improvement. Currently, we assume that providers act fairly, and we do not expect to have injected or maliciously manipulated event logs. In addition, we do not handle TEE crashes and suppose that miners and providers exchange messages in perfect communication channels where no loss, no snap, and no bit corruption take place. %We also make assumptions on event log data. Our solution presupposes a universal clock for event timestamps in different systems; hence no synchronization procedure is actually addressed. Moreover, traces of different organizations referring to the same process instance are always characterized by the same case identifier. This is implausible in real-world scenarios, and organizations may adopt different case notations. To overcome this issue, alternative event log representation should be considered.
We also make assumptions on event log data. We assume the existence of a universal clock for event timestamps across various systems, eliminating the need for synchronization procedures. Additionally, we presume that traces from different organizations relating to the same process instance share a common case identifier. However, this assumption is unrealistic in real-world scenarios, where organizations might employ different case notations. To address this challenge, we should explore alternative event log representations. Future work includes the elaboration of an interaction protocol that formalizes the communication workflow between data providers and miners. Additionally, we plan to integrate usage control policies containing terms and conditions on event log utilization. To achieve this goal, we will design dedicated mechanisms inside trusted applications for monitoring usage rules and enforcing their fulfillment. The presented solution embraces model process mining techniques in a general way. However, we believe that the presented approach is particularly compatible with declarative model representations. Therefore, trusted applications could compute and store the entire set of rules representing a business process, and users may interact with them via trusted queries. We plan to extend the discussion in \cref{sec:evaluation} by integrating threat modeling analysis and quantitative assessments concerning scalability, throughput, and performances on real-world event logs.
\todo{CDC: The bibliography entries are too rich. Look at \cite{engel2011process}. Do we really care that the conference was in Toulouse? And look at \cite{koch2002expressive}: the acronym is enough for the conference name. Also, the volume number is useless if we do not have the series (anyway, we could not care less about either of the two). We have already gone through this, so we should shorten the entries as we know.}

\begin{comment}
Limitations:
\begin{itemize}
    \item Both producer and consumer act fairly (so we do not expect to have injected data) ok
    \item We do not manage TEE crashes
    \item We assume a perfect communication channel (no loss, no snap, no corrupted bits)
    \item Universal clock for event timestamps (cite Event log cleaning for business process analytics by Andreas Solti)
\end{itemize} 
Future Work:
\begin{itemize}
    \item Declarative models adaptation ok
    \item Output inside the TEE, interactions through trusted applications
    \item Real-world event log data ok
    \item Usage policies integration ok
    \item Formal interaction protocol ok
    \item Threat model ok
    \item Security evaluation ok
\end{itemize}
\end{comment}
%future work, quali segment portano al risultato attuale e farlo per ogni risultato intermedio -call valerio claudio 9-06-23
