\section{Related Work}
\label{sec:background}
The theme of inter-organizational process mining is discussed in the literature from different perspectives. The work of M{\"u}ller et al.~\cite{muller2021process} was one of the first to pay attention to data privacy and security within third-party systems that mine other activities' data. In order to preserve the information to be mined, this research proposes a conceptual architecture in which process mining algorithms are executed inside cloud service equipped with trusted execution environment. Inspired by this preliminary contribution, we design an approach where each organization can run process mining algorithms without involving external stakeholders. Unlike M{\"u}ller et al. work in which an algorithm executed in the cloud sends the same result to all the organizations in the collaboration environment, in our architecture each organization is autonomous to choose when performing the mining operations.
Elkoumy et al.~\cite{elkoumy2020shareprom,elkoumy2020secure} present a tool called Shareprom. Shareprom allows independent parties to perform mining operations in inter-company contexts without revealing their input data to other parties included in the context. Like our work Shareprom aims to protect the data of the companies involved in the mining operation. Shareprom is only capable of performing operations with directed acyclic graphs that are exchanged in a protected manner between parties. Unlike our work, where exchanged data are company logs. Using this type of graph restricts the possible use of Shareprom in many contexts, although they are widely used as process representations in process mining, other types of data or representations may be needed in many process mining contexts. In addition, the technology used by Shareprom is secure multiparty computation which does not guarantee high scalability. Our work solves this problem by using trusted applications that execute inside trusted execution environments owned by all parties involved in the inter-organizational context. The results obtained from out work on scalability will be shown in the discussion section.









Analyzing inter-organizational business processes: process mining and business performance analysis using electronic data interchange messages ~\cite{engel2016analyzing}

Merging Event Logs for Inter-organizational Process Mining~\cite{hernandez2021merging}

Merging event logs for process mining: A rule based merging method and rule suggestion algorithm~\cite{claes2014merging}



























































%%%%%%%%%%%%%%%%%%%%%%%%%%%%%%%%%%%%%%%%%%%%%%%%%%%%%%%%%%%%%%%%%%%%%%%%%%%
%%%%%%%%%%%%%%%%%%%%%%%%%%%%%%%%%%%%%%%%%%%%%%%%%%%%%%%%%%%%%%%%%%%%%%%%%%%
%%%%%%%%%%%%%%%%%%%%%%%%%%%%%%%%%%%%%%%%%%%%%%%%%%%%%%%%%%%%%%%%%%%%%%%%%%%
%%%%%%%%%%%%%%%%%%%%%%%%%%%%%%%%OLD%%%%%%%%%%%%%%%%%%%%%%%%%%%%%%%%%%%%%%%%
%%%%%%%%%%%%%%%%%%%%%%%%%%%%%%%%%%%%%%%%%%%%%%%%%%%%%%%%%%%%%%%%%%%%%%%%%%%
%%%%%%%%%%%%%%%%%%%%%%%%%%%%%%%%%%%%%%%%%%%%%%%%%%%%%%%%%%%%%%%%%%%%%%%%%%%
%%%%%%%%%%%%%%%%%%%%%%%%%%%%%%%%%%%%%%%%%%%%%%%%%%%%%%%%%%%%%%%%%%%%%%%%%%%
\begin{comment}
The work of M{\"u}ller et al.~\cite{muller2021process} is the first contribution that considers TEEs in combination with blockchain technologies for process mining purposes. This research proposes a conceptual architecture in which process mining algorithms are executed inside centralized third-party services. Inspired by this preliminary contribution, we design a decentralized approach context where each organization can run process mining algorithms without involving external stakeholders.
The literature proposes several studies that consider process mining techniques in inter-organizational environments. Van Der Aalst~\cite{van2011intra} shows that inter-organizational processes can be divided according to different dimensions making identifiable challenges of inter-organizational process extractions. Elkoumy et al.~\cite{elkoumy2020shareprom} propose a tool that allows independent parts of an organization to perform process mining operations by revealing only the result. This tool is called Shareprom and exploits the features of secure multi-party computation (MPC). Engel et al.~\cite{engel2016analyzing} present EDImine Framework, which allows to apply process mining operations for inter-organizational processes supported by the EDI standard\footnote{https://edicomgroup.com/learning-center/edi/standards} and evaluate their performance using business information.
Elkoumy et al.~\cite{elkoumy2020secure} propose an MPC-based architecture that aims to perform process mining operations without sharing their data or trusting third parties.
% inter-organizational and merge log
Applying process mining techniques in intra-organizational contexts requires merging the event logs of the organizations participating in the process. The literature offers several studies in this area. For instance, Hernandez-Resendiz et al.~\cite{hernandez2021merging} present a methodology for merging logs at the trace and activity level using rules and methods to discover the process. Claes et al.~\cite{claes2014merging} provide techniques for performing merge operations in inter-organizational environments. This paper indicates rules for merging data in order to perform process mining algorithms.
%data exchange 
The state of the art provides some studies that investigate issues and possible solutions regarding data exchange, more specifically in an business collaboration context. EDI standards enable the communication of business documents. Among these standards, the notion of process is not explicitly specified. This inhibits organizations from applying Business Process Management (BPM) methods in business collaboration environments. Engel et al.\cite{engel2011process} extended process mining techniques by discovering interaction sequences between business partners based on EDI exchanged documents. Lo et al.\cite{lo2020flexible} have provided and developed a framework for data exchange designed even in intra-organizational situations. This framework is based on blockchain and decentralized public key infrastructure technologies, which ensure scalability, reliability, data security, and data privacy.
% Use of data from other organizations(or person) integrity ecc ecc (LOCAL)
Additionally, there are several papers that propose solutions for the correct sharing and use of data by third parties. Xie et al.\cite{XIE2023321} propose an architecture for the internet of things based on TEE and blockchain. The proposed architecture aims to solve data and identity security problems in the process of data sharing. Basile et al.~\cite{Basile_Blockchain_based_resource_governance_for_decentralized_web_environments} in their study created a framework called ReGov that allows the exchange of sensitive information in a decentralized web context, ensuring usage control-based data access and usage. In order to control the consumer's device ReGov uses TEE that allows storage and utilization management of retrieved resources. Hussain et al.\cite{hussain2021sharing} present a tool for privacy protection and data management among multiple collaborating companies. This tool allows data encryption to be configured according to the privacy obligations dictated by the context of a system's use. 



\todo[inline]{It can be reduced. EDIT: Already reduced. MISSING: what do we do similarly to and what do we do differently from / improve on the cited papers? A comparison is offered only with the work of M{\"u}ller et al.}
\todo[inline]{Our work revolves around the following areas: 1, 2 and 3. Next, we position our contribution against the existing body of literature.}
\todo{When do we use blockchain in this paper? If we do not use it, then we should make clear why we do not. Isn't it also the first paper that uses TEE for process mining? If it is, then we omit this blockchain thing and take it back for future work perhaps or as an additional detail.}
\todo[inline]{Our work revolves around the following areas: 1, 2 and 3. Next, we position our contribution against the existing body of literature.}
The theme of inter-organizational process mining is discussed in the literature from different perspectives. The work of M{\"u}ller et al.~\cite{muller2021process} is the first contribution that considers TEEs in combination with %blockchain technologies for process mining purposes.
process mining techniques.
% \todo{When do we use blockchain in this paper? If we do not use it, then we should make clear why we do not. Isn't it also the first paper that uses TEE for process mining? If it is, then we omit this blockchain thing and take it back for future work perhaps or as an additional detail.}
This research proposes a conceptual architecture in which process mining algorithms are executed inside centralized third-party services. Inspired by this preliminary contribution, we design a decentralized approach context where each organization can run process mining algorithms without involving external stakeholders. 
Another approach in the field of inter-organizational process mining is presented by Elkoumy et al.~\cite{elkoumy2020shareprom} with Shareprom. In contrast with our work, Shareprom leverages multi-party computation to combine event log data spread across cooperating parties and synchronize the execution of process mining algorithms.
In addition, the work done by Engel et al.~\cite{engel2016analyzing} present the EDImine framework, which applies process mining to inter-organizational processes focusing on the EDI standard, contrastingly to our work which is designed considering the XES standard. In inter-organizational contexts, merging event logs from different organizations is essential. In our work to merge logs from organizations, we were inspired by the following works. Hernandez-Resendiz et al.~\cite{hernandez2021merging} propose a methodology for log merging, while Claes et al.~\cite{claes2014merging} provide techniques for merging data to support process mining algorithms. In order to merge data, organizations must be able to exchange information with each other in a secure manner. Lo et al.~\cite{lo2020flexible} present a blockchain-based framework for secure data exchange, even within inter-organizational scenarios. Our work differs from previous work because the exchange of data is done through the use of trusted parties which communicates with other organizations and preserves data. Lastly, there are solutions for secure data sharing with third parties. Xie et al.~\cite{XIE2023321} propose an IoT architecture using TEE and blockchain. Basile et al.~\cite{Basile_Blockchain_based_resource_governance_for_decentralized_web_environments} introduce ReGov for controlled data utilization in decentralized web contexts. Our work has as its point of contact with previous work the use of tee technologies. However, it differs by not considering a blockchain that orchestrates collaboration between organizations. % Hussain et al.~\cite{hussain2021sharing} offers a tool for privacy protection and data management in collaborative settings, allowing data encryption to align with privacy requirements.
\end{comment}