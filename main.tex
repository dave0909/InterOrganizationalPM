% This is samplepaper.tex, a sample chapter demonstrating the
% LLNCS macro package for Springer Computer Science proceedings;
% Version 2.20 of 2017/10/04
%
\documentclass[runningheads]{llncs}
%
% Used for displaying a sample figure. If possible, figure files should
% be included in EPS format.
% For language-specific hyphenations etc.
\usepackage[english]{babel}
% For subfigures
\usepackage{subcaption}
\usepackage{xcolor}
% For nice links
\usepackage{url}
% For playing with colors in tabular environments
\usepackage{colortbl}
% For math symbols, such as \nexists
\usepackage{amssymb}
% For more math symbols, such as \mapsfrom
\usepackage{stmaryrd}
% For advanced graphics
% For equations, arrays of equations, defining operator names, etc.
\usepackage{amsmath}
% For cursive math
\usepackage{mathrsfs}
% For math symbols, such as \nexists
\usepackage{amssymb}
%% For math environments, such as "definition"
%\usepackage{amsthm}
%\theoremstyle{definition}
%\newdefinition{definition}{Definition}[section]
%\newtheorem{theorem}{Theorem}[section]
% For enumerating the line numbers
\usepackage[left]{lineno}
% For diagonally-split fractions (\sfrac)
\usepackage{xfrac}
% For nice (diagonal) fractions
\usepackage{nicefrac}
% For side notes, missing figures and inline to-do's
\usepackage[textsize=scriptsize,backgroundcolor=yellow!40]{todonotes}
% Resize the width of todo-notes on the margins
%\setlength{\marginparwidth}{1.25cm}
% For specifying kewords and acronyms
\usepackage[nonumberlist,acronym,sanitize=none]{glossaries}
\glsdisablehyper
% For commenting out some parts of the text
\usepackage{comment}
% For hyperlinks
\usepackage[pdftex, colorlinks=true, hyperfootnotes=true, hyperindex=true,
            plainpages=false, pagebackref=false, pdfpagelabels=true, pdfstartview=FitH,
            linkcolor=blue, citecolor=blue, urlcolor=blue,
            bookmarks, bookmarksopen, bookmarksdepth=3]{hyperref}
% For smart references
\usepackage[capitalise,nameinlink]{cleveref}
% To have "Figure 3(a)" in place of "Figure 3a" and  "Table 3(a)" in place of "Table 3a"
\captionsetup[subfigure]{subrefformat=simple,labelformat=simple}
    \renewcommand\thesubfigure{(\alph{subfigure})}
\captionsetup[subtable]{subrefformat=simple,labelformat=simple}
    \renewcommand\thesubtable{(\alph{subtable})}
\crefname{algocf}{alg.}{algs.}
\Crefname{algocf}{Algorithm}{Algorithms}
\crefname{section}{Sect.}{Sects.} % Do not use too many char's
\crefname{Section}{Section}{Sections}
% TikZ/Pgf advanced graphics
\usepackage{tkz-base}
\usetikzlibrary{decorations.pathmorphing,trees,snakes,arrows,shapes,automata,petri}
% To use inline and other fancy list-like environments (e.g., inparaenum)
\usepackage{paralist}
% To divide a text line into multiple columns
\usepackage{multicol}
% To create good-looking book-style tables
\usepackage{booktabs}
% To play around with list environments
\usepackage{enumitem}
% To create multirow cells in tables
\usepackage{multirow}
% To create rotated cells in tables
\usepackage{rotating}
% To create enumerated lists, whose numbering is reversed
\usepackage{etaremune}
%% To make algorithmic nice-looking pseudocode
\usepackage[ruled,linesnumbered,algo2e]{algorithm2e}
%% For creating side-notes
\usepackage{marginnote}
% For superimposing symbols over one another within math env.
\usepackage{mathtools}
% For strange math symbols like \Dashv
\usepackage{mathabx}
% For adjusting the size of tables to the text width, if necessary: \begin{adjustbox}{max width=\textwidth}
\usepackage{adjustbox}
% For LaTeX if/then statements
\usepackage{ifthen}
% For strike-through cancellations
\usepackage[normalem]{ulem}
%% The lineno packages adds line numbers. Start line numbering with
%% \begin{linenumbers}, end it with \end{linenumbers}. Or switch it on
%% for the whole article with \linenumbers.
\usepackage{lineno}
% For highlighted text
\usepackage{soul}
% To put table environments and co. side by side
\usepackage{floatrow}
\floatsetup[table]{style=plaintop}
\usepackage{listings}
\begin{comment}
	contenuto...
\lstset{ %
%	backgroundcolor=\color{white},   % choose the background color; you must add \usepackage{color} or \usepackage{xcolor}; should come as last argument
	basicstyle=\tiny\ttfamily,       % the size of the fonts that are used for the code
	breakatwhitespace=true,          % sets if automatic breaks should only happen at whitespace
	breaklines=true,                 % sets automatic line breaking
	captionpos=b,                    % sets the caption-position to bottom
	commentstyle=\color{gray},       % comment style
%	escapeinside={\%*}{*)},          % if you want to add LaTeX within your code
	extendedchars=true,              % lets you use non-ASCII characters; for 8-bits encodings only, does not work with UTF-8
	frame=single,	                 % adds a frame around the code
	keepspaces=true,                 % keeps spaces in text, useful for keeping indentation of code (possibly needs columns=flexible)
%	keywordstyle=\color{blue},       % keyword style
%	language=SQL,                    % the language of the code
%	deletekeywords={Time},      % if you want to delete keywords from the given language
%	morekeywords={*,...},            % if you want to add more keywords to the set
	numbers=left,                    % where to put the line-numbers; possible values are (none, left, right)
	numbersep=5pt,                   % how far the line-numbers are from the code
	numberstyle=\tiny\color{gray}, % the style that is used for the line-numbers
%	rulecolor=\color{black},         % if not set, the frame-color may be changed on line-breaks within not-black text (e.g. comments (green here))
	showspaces=false,                % show spaces everywhere adding particular underscores; it overrides 'showstringspaces'
	showstringspaces=false,          % underline spaces within strings only
	showtabs=false,                  % show tabs within strings adding particular underscores
	stepnumber=1,                    % the step between two line-numbers. If it's 1, each line will be numbered
%	stringstyle=\color{mymauve},     % string literal style
	tabsize=2,	                   % sets default tabsize to 2 spaces
%	title=\lstname                   % show the filename of files included with \lstinputlisting; also try caption instead of title
	morecomment=[l]\%
}
\end{comment}

\lstset{
	backgroundcolor=\color{scriptcolor},
	extendedchars=true,
	basicstyle=\fontsize{8pt}{8pt}\selectfont\ttfamily,
	showstringspaces=false,
	showspaces=false,
%	linewidth=19em,
	numbers=left,
	numberstyle=\footnotesize,
	numbersep=5pt,
	tabsize=2,
	breaklines=true,
	showtabs=false,
	captionpos=b,
	%belowskip=-0.2em,
	lineskip = 0.1em
}
\crefname{lstlisting}{Listing}{Listings}
% To add dummy text
\usepackage{lipsum}
% To have newlines in cells, with commands such as \makecell or \thead
\usepackage{makecell}
% For diagonal lines in tables
\usepackage{diagbox}
% For a decent formatting of numbers (and a wonderful system for numeric columns in tables, ``S'')
\usepackage[scientific-notation=false,group-separator={,}]{siunitx}
% To enable text protrusion
\usepackage{microtype}
% For footnote-references: \footref
\usepackage{footmisc}
% For side figures
\usepackage{wrapfig}
% To create nice ORCiD-like circles in the authors' block
\usepackage{orcidlink}
% Override casing of text (e.g., via \MakeTextLowercase
\usepackage{textcase}
% Override the writing of dates (here, 09/01/2024 rather than January 9, 2024)
\usepackage[ddmmyyyy]{datetime}
%Override line space in pseudocode
\usepackage{setspace}

% If you use the hyperref package, please uncomment the following line
% to display URLs in blue roman font according to Springer's eBook style:
\renewcommand\UrlFont{\color{blue}\rmfamily}
%
\begin{document}
%
\title{
}
%
%\titlerunning{Abbreviated paper title}
% If the paper title is too long for the running head, you can set
% an abbreviated paper title here
%
\author{Davide~Basile\inst{1}\orcidID{0000-1111-2222-3333} \and
Luca~Barbaro\inst{1}\orcidID{0000-0002-2975-5330} \and
Valerio~Goretti\inst{1}\orcidID{0000-0001-9714-4278} \and Claudio~Di~Ciccio\inst{1}\orcidID{2222--3333-4444-5555}}
%
\authorrunning{F. Author et al.}
% First names are abbreviated in the running head.
% If there are more than two authors, 'et al.' is used.
%
\institute{Sapienza University of Rome}
%
\title{Trusted Applications for Inter-organizational Process Mining}
\maketitle

\begin{abstract}
 Through process mining tecniques, organizations enhances their operational efficiency, improve performances, and deepen the understanding of their business processes. While most process mining research focuses on intra-organizational settings, the emerging importance of inter-organizational collaborations for operational excellence cannot be ignored. Inter-organizational business processes involve multiple independent organizations collaborating to achieve mutual interests. However, inter-organizational process mining faces substantial challenges, primarily centered on confidentiality concerns. In this paper, we introduce a novel approach based on the adoption of trusted applications running in Trusted Execution Environments (TEEs). Our research work aims at ensuring privacy preservation and safeguarding the integrity of sensitive information during process mining procedures in inter-organizational contexts. To achieve this, we introduce a TEE-based infrastructure supporting the execution of trusted applications through which partner organizations securely share operational information and apply process mining tecniques. We show the feasibility of our solution by considering an healthcare scenario that serve as running example. Our contribution include a discussion of the proposed research work that addresses strenghts and areas for improvement.
\end{abstract}



\section{Introduction}
\label{sec:introduction}
\begin{comment}
\todo[inline]{%
	´CDC: Status corrente secondo me. Abbiamo\ldots
	Un'intro affabile ma che dobbiamo ricalibrare per toglierla dal pantano della TEE e puntare dritto al nostro obiettivo dichiarato e mantenuto, con vista sul potenziale inespresso.\\
	Un motivating scenario che però resta staccato dal resto.\\
	Un related work che però dovrebbe spostarsi sul far capire cosa facciamo di più e di diverso, o di simile ma rivisto.\\
	Una overview eccellente dell'architettura che però non è connessa con l'esempio.\\
	Una overview accurata delle interazioni dei moduli (e quando lo spazio è poco è il caso di fare il merge tra le due cose per risparmiare spazio, come ci dicemmo) -- anche qui senza esempio, dunque capire cosa si fa dove è arduo. Non per me o per te, ma perché conosciamo il lavoro. Chi non lo conosce, non ha assolutamente idea di cosa voglia dire fare il merge degli eventi da tracce provienienti da actor diversi rimettendoli in ordine così da preservare integrità temporale, per esempio.\\
	Una discussion/evaluation ben congegnata in cui però manca la parte quantitativa e la qualitativa è ancora da dettagliare.\\
	Una conclusione.%
}
\end{comment}
In today's business landscape, organizations constantly seek ways to enhance operational efficiency, increase performance, and gain valuable insights to improve their processes. Process mining offers techniques to discover, monitor, and improve business processes by extracting knowledge from chronological records known as \textit{event logs}. Organizations record in these ledgers events referring to activities and interactions occurring within a business process. The vast majority of process mining contributions consider \textit{intra-organizational} settings, in which business processes are executed inside individual organizations. However, organizations increasingly recognize the value of collaboration and synergy in achieving operational excellence. \textit{Inter-organizational} business processes involve several independent organizations actively cooperating to achieve a shared objective. Despite the advantages in terms of transparency, performance optimization, and benchmarking that companies can gain from such practices, inter-organizational process mining raises challenges that make it still hardly applicable. The major issue concerns confidentiality. Companies are reluctant to outsource to their partners inside information that is required to execute process mining algorithms. Indeed, the sharing of sensitive operational data across organizational boundaries introduces concerns about data privacy, security, and compliance with regulations. \textit{Trusted Execution Environments} (TEEs) can serve as fundamental enablers to balance the need for insights with the imperative to protect sensitive information in inter-organizational settings. TEEs offer secure contexts that guarantee code integrity and data confidentiality in external devices. \textit{Trusted applications} are tamper-proof software objects running in these environments. 

In this paper, we propose a novel approach for inter-organizational process mining that resorts to trusted applications to preserve the secrecy and integrity of shared data. To pursue this aim, we design a decentralized software architecture for a three-staged procedure: (i) the initial exchange of preliminary metadata (ii) the secure transmission of encrypted data amid multiple parties, (iii) the privacy-preserving merge of the shared information segments followed by the isolated and verifiable computation of process discovery algorithms on joined data.
We evaluate our proof-of-concept implementation against synthetic and real-world-based data with a convergence test and memory effectiveness assessment.

The remainder of the paper is structured as follows: \cref{sec:background} provides an overview of related work inherent to the theme of inter-organizational process mining. In \cref{sec:motivating}, we introduce a use case example that considers a healthcare scenario. The high-level architecture of our solution is presented in \cref{sec:design}. Following on from this, we instantiate the addressed design principles in \cref{sec:realization} focusing on the employed technologies, workflow, and implementation. In \cref{sec:evaluation}, we discuss our solution. Finally, we conclude and present directions for future work in \cref{sec:conclusion}.

\section{Related Work}
\label{sec:background}

% inter-organizational and process mining
The literature proposes several studies that consider process mining techniques on inter-organizational environments.  Van Der Aalst~\cite{van2011intra} shows that inter-organizational processes can be divided according to different dimensions making identifiable challenges of inter-organizational process extractions. Elkoumy et al.~\cite{elkoumy2020shareprom} propose a tool which allows independent parts of an organization to perform process mining operations by revealing only the result. This tool is called Shareprom and exploits the features of secure multi-party computation (MPC). Engel et al.~\cite{engel2016analyzing}
present EDImine Framework which allows to apply process mining operations for inter-organizational processes supported by the EDI standard\footnote{https://edicomgroup.com/learning-center/edi/standards} and evaluate their performance using business information.
Elkoumy et al.~\cite{elkoumy2020secure} They propose an architecture based on MPC. This architecture aims to perform process mining operations without the need to share their data or trust third parties.



% inter-organizational and merge log
\cite{hernandez2021merging}
\cite{claes2014merging}
% data exchange
% Use of data from other organizations(person) integrity ecc ecc (LOCAL)
Once the data exchange has taken place, it is critical that the data be stored in a trusted part of the consumer's device. Basile et al.~\cite{Basile_Blockchain_based_resource_governance_for_decentralized_web_environments} in their study created a framework called Regov that allows for the exchange of sensitive information in a decentralized web context, ensuring usage control-based data access and usage. To ensure control over the consumer's device, Davide et al. use trusted execution environment that allows storage and utilization management of retrieved resources. 

The application of process mining in an inter-organizational scenario is infrequent due to concerns related to privacy and confidentiality, integrity and data heterogeneity. To overcome these problems, a large number of techniques have been proposed. Federated process mining \cite{van2021federated} aims to effectively manage the problem of privacy-preserving. Using federated data sources, event data can be transparently mapped between multi-source autonomous provider to monitor, analyze, and improve processes across organizations.


\section{Motivating Scenario}\label{sec:motivating}
%In the fast-evolving landscape of healthcare, seamless collaboration between multiple organizations is essential to ensure the highest standard of patient care. We delve into the application of Trusted Execution Environment (TEE) to facilitate the secure exchange of event logs between three pivotal actors: an esteemed hospital, a specialized clinic, and a leading pharmaceutical company. This innovative approach fosters a robust and trustworthy ecosystem where sensitive patient data can be shared securely, promoting seamless collaboration for the betterment of patient outcomes.

%Sintetizza il processo tra ospedale azienda farmaceutica e struttura specializzata
\begin{figure}[t]
\centering
\includegraphics[width=0.9\linewidth]{content/figures/healthcarescenario.pdf}
\caption{BPMN Healthcare Scenario}
\label{fig:BPMN_Healthcare}
\end{figure}

\begin{table}[t]
\begin{tabular}{|lll|l|lll|l|lll|}
\cline{1-3} \cline{5-7} \cline{9-11}
\multicolumn{3}{|l|}{Hospital}                                               &  & \multicolumn{3}{l|}{Pharmaceutical company}                                 &  & \multicolumn{3}{l|}{Specialised Clinic}                                     \\ \cline{1-3} \cline{5-7} \cline{9-11} 
\multicolumn{1}{|l|}{Trace id} & \multicolumn{1}{l|}{Event name} & Timestamp &  & \multicolumn{1}{l|}{Trace id} & \multicolumn{1}{l|}{Event name} & Timestamp &  & \multicolumn{1}{l|}{Trace id} & \multicolumn{1}{l|}{Event name} & Timestamp \\ \cline{1-3} \cline{5-7} \cline{9-11} 
\multicolumn{1}{|l|}{}         & \multicolumn{1}{l|}{}           &           &  & \multicolumn{1}{l|}{}         & \multicolumn{1}{l|}{}           &           &  & \multicolumn{1}{l|}{}         & \multicolumn{1}{l|}{}           &           \\ \cline{1-3} \cline{5-7} \cline{9-11} 
\multicolumn{1}{|l|}{}         & \multicolumn{1}{l|}{}           &           &  & \multicolumn{1}{l|}{}         & \multicolumn{1}{l|}{}           &           &  & \multicolumn{1}{l|}{}         & \multicolumn{1}{l|}{}           &           \\ \cline{1-3} \cline{5-7} \cline{9-11} 
\end{tabular}
\end{table}

\begin{table}[t]
\begin{tabular}{lll}
\hline
\multicolumn{3}{|l|}{Hospital}                                                                             \\ \hline
\multicolumn{1}{|l|}{Trace id} & \multicolumn{2}{l|}{Trace}                                                \\ \hline
\multicolumn{1}{|l|}{0}        & \multicolumn{2}{l|}{\textless CPA, OD, RD, AD, PRM \textgreater{}}        \\ \hline
                               &                                      &                                    \\ \hline
\multicolumn{3}{|l|}{Pharmaceutical Company}                                                               \\ \hline
\multicolumn{1}{|l|}{Trace id} & \multicolumn{2}{l|}{Trace}                                                \\ \hline
\multicolumn{1}{|l|}{0}        & \multicolumn{2}{l|}{\textless BA, TI, MÒ, GI \textgreater{}}              \\ \hline
                               &                                      &                                    \\ \hline
\multicolumn{3}{|l|}{Specialised Clinic}                                                                   \\ \hline
\multicolumn{1}{|l|}{Trace id} & \multicolumn{2}{l|}{Trace}                                                \\ \hline
\multicolumn{1}{|l|}{0}        & \multicolumn{2}{l|}{\textless{}CT, QE, BTL, ASM, RTB, CDO \textgreater{}} \\ \hline
\end{tabular}
\end{table}

\section{Motivating Scenario}\label{sec:motivating}
Cooperation between different structures is crucial in the medical field, and many processes are frequently outsourced. In our running example, we consider a simplified healthcare scenario in which an hospitalization process involves the cooperation of three organizations, namely, the \texttt{Hospital}, the \texttt{Pharmaceutical Organization}, and the \texttt{Specialized Clinic}. We explain the process depicted in the BPMN diagram in \cref{fig:BPMN_Healthcare} as follows.
Preliminary examinations are carried out when a patient enters the \texttt{Hospital}. Then, the \texttt{Hospital} orders the drugs needed to treat the patient from the \texttt{Pharmaceutical Company}. The \texttt{Pharmaceutical Company} receives the order, produces the drugs in the laboratory, and sends them back to the \texttt{Hospital}. The \texttt{Hospital} manages the drugs received and verifies whether the patient can be treated internally. Patients requiring special care are transferred to the \texttt{Specialized Clinic} where more in-depth checks are carried out. Once the \texttt{Specialized Clinic} has verified the response to the alternative treatment, the patient is transferred to the \texttt{Hospital} to prepare the dehospitalization. Before discharging the patient, the \texttt{Hospital} prepares the necessary clinical documentation. In addition, the \texttt{Hospital} also carries out analysis checks. Finally, the \texttt{Hospital} discharges the patient.

The \texttt{Hospital}, the  organizations record the events belonging to different process instances in the traces of their event logs. In \cref{fig:BPMN_Healthcare}, we propose as an example two possible traces scattered across the three organizations. Traces of different organizations having the same ID are in the same process instance. TO BE CONTINUED ONCE WE HAVE THE IMAGE.\todo[]{Add two trace examples below the bpmn.}

%Any organization in the collaboration environment can request to perform process mining operations. For instance, the hospital cooperates with the specialized clinic and the pharmaceutical company, it can decide at any time to analyze the entire process by considering data from all three partners to provide an overview. The hospital requests the necessary data from all companies participating in the cooperation. All companies will send their process data in event log form. To mine all data together, the hospital must merge the event logs received. Once the event log has been merged, the hospital can proceed with the execution of the mining algorithm to analyze the entire process.

%TRACE 555

%Patient hospitalised, Carry out preliminary analyses , Order drugs, Receive drugs, Administer drug, Transfer the patient to chosen specialised clinic, Receive the patient back from specialised clinic, Declare patient healed, Prepare clinic documentation, Discharge patient

% ()Receive drugs order from hospital (2022-12-15T09:30:00Z), Produce new drugs in laboratory (2022-12-15T11:30:00Z),  Ship drugs to hospital (2022-12-15T13:30:00Z)

% Specialised: Patient arrives from hospital (2022-12-16T00:30:00Z), Perform more in-depth analyses (2022-12-16T02:30:00Z), Perform treatment (2022-12-16T04:30:00Z), Verify response to alternative treatment (2022-12-16T05:30:00Z), Transfer patient back to hospital (2022-12-16T06:30:00Z)

\section{Design}\label{sec:design}
\section{High level Architecture}
In the following section we present the high level architecture underlying our solution. Therefore, we take into account each component individually. Once introduced the architecture, we provide an overview on the main interactions taking place between the introduced components.
\subsection{Main components}
\label{fig:architecture_diagram}
\begin{figure}[t]
\centering
\includegraphics[width=10cm]{content/figures/architecture_diagram.pdf}
\caption{High-level architectural overview of the framework.}
\label{fig:implementation}
\end{figure}
Our architecture involve networks of nodes controlled by different \texttt{Organization}s exchanging their event logs. \texttt{Organization}s in the same network collaborate to reach a common objective sharing one or more business processes. The Hospital and Specialized Clinic, mentioned in the running example, provide an example of partner organizations.

In \cref{fig:architecture_diagram}, we propose an high level schematization of our solution. Each organization embeds three main components: the \texttt{EMS Interface}, the \texttt{Log Provider} and the \texttt{Trusted Miner}. %collects the logic to interact with the Environmental Management System (EMS) of the organization. The \texttt{Log Provider} component deliver on-demand data to partner organization's systems. The \texttt{Trusted Miner} executes process mining algorithms inside the \texttt{Trusted Execution Environment} using event log retrieved from partner organizations.
In the next paragraph, we address the newly mentioned components.
\subsubsection{ERP Interface}
The \texttt{ERP Interface} collects the logic to interact with the Enterprise Resource Planning (ERP) system  of the \texttt{Organization}. ERP systems helps organizations to
handle business processes including accounting and resource management. The mantainance of event logs is one of the many tasks performed by these systems. In our architecture, we generalize the interaction with this systems through the \texttt{ERP Interface}. The \texttt{ERP Interface} provide the local \texttt{Log Provider} and \texttt{Trusted Miner} a direct way to access event logs generated inside the organization.

\subsubsection{Log Provider}
The \texttt{Log Provider} component deliver on-demand data to partner organization's systems.
\subsubsection{Trusted Miner}
 The \texttt{Trusted Miner} executes process mining algorithms inside the \texttt{Trusted Execution Environment} using local event log and event log retrieved from partner organizations.
\subsection{Workflow}
\subsubsection{Initialization}
\subsubsection{Data Exchange}
\subsubsection{Data Elaboration}




\section{Realization}
%Focus generale sulle tecnologie utilizzate
In this section we outline the technical aspects concerning the realization of our framework. Therefore we first present the enabler technologies through which we instantiate the design principles presented in \cref{sec:design}. After that, we discuss the interaction workflow between the instantiated technologies. Finally, we show the implementation details.
\begin{figure}[t]
\centering
\includegraphics[width=11cm]{content/figures/deployment_diagram.pdf}
\caption{UML deployment diagram.}
\label{fig:deployment_diagram}
\end{figure}
\subsection{Deployment}
As follow, we bridge the gap between high-level system architecture and its practical realization. \cref{fig:deployment_diagram} depicts a \textit{UML deployment diagram} \cite{koch2002expressive} aimed at aiding the understanding of the instantiated infrastrucuture. 

The \texttt{Organization Machine} represent the physical computation \textit{device} embracing the software and hardware entities of the company. The \texttt{PAIS}, the \texttt{PAIS Interface}, the \texttt{Log Provider} and \texttt{Secure Miner} are included in the \texttt{Organization Machine} as abstract \textit{components} . These elements serve as logical containers that incorporate the core functionalities already discussed in \cref{sec:design}. The \texttt{Organization Machine} is characterized by two \textit{execution environment}s: The \texttt{Operative System} and the \texttt{Trusted Execution Environment}.

\subsection{Workflow}
\subsection{Implementation}
%\subsubsection{Event Log Generation}
%Tecnologie utilizzate
%Sintesi del processo di generazione dei log
%\subsubsection{Trusted Miner and Log Provider}
%Tecnologia TEE usata
%Linguaggio usato per programmare in TEE
%Algoritmo implementato
%Rappresentazioni intermedie (PNML, Petrinet, ecc...)
%Linguaggio Log provider

\section{Evaluation}


%\subsection{Validation: Volvo}
\subsection{Convergence study}
\subsubsection{Settings}
\subsubsection{Results}

\section{Conclusion and Future Work}
\label{sec:conclusion}
Limitations:
\begin{itemize}
    \item Both producer and consumer act fairly (so we do not expect to have injected data)
    \item We do not manage TEE crashes
    \item We assume a perfect communication channel (no loss, no snap, no corrupted bits)
    \item Universal clock for event timestamps (cite Event log cleaning for business process analytics by Andreas Solti)
\end{itemize} 
Future Work:
\begin{itemize}
    \item Declarative models adaptation
    \item Output inside the TEE, interactions through trusted applications
    \item Real-world event log data
    \item Usage policies integration
    \item Formal interaction protocol
    \item Threat model
    \item Security evaluation
\end{itemize}


%future work, quali segment portano al risultato attuale e farlo per ogni risultato intermedio -call valerio claudio 9-06-23

%%%%%%%%%%%%%%%%%%%%%%%%%%%%%%%%%%%%%%%%%%%%%
% Biblography
%%%%%%%%%%%%%%%%%%%%%%%%%%%%%%%%%%%%%%%%%%%%%
\bibliographystyle{splncs04}
\bibliography{main}


\end{document}
