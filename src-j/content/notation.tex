\begin{newj}
Thus far, we have intuitively introduced the concepts of events, (partial) cases, (partial) traces, event logs and related aspects.
In this section, we delve deeper into these notions from a formal standpoint. The definitions provided henceforward pave the ground for properties and aspects to which we will resort in the design and realization of our solution.

\begin{definition}[Event]\label{def:evt}
	Let $\EvtU \ni \Evt$ be a finite non-empty set of symbols. We refer to $\Evt$ as \emph{event} and to $\EvtU$ as the \emph{universe of events}.
	An \emph{attribute} is a function having the universe of events as its domain.
\end{definition}
%
%In the example, $\Evt=\{\texttt{312},\ \texttt{2022-07-14T10:36},\ \texttt{PH}\}$ represents the event occurring when Alice's hospitalization journey begins. An attribute is a generic function that returns one or more symbols of $\Evt$.\todo{Add example.Done}
Referring to the example of Alice and Bob's hospitalization process (\cref{sec:motivating}), each event denoted as $\Evt_i$ with $1 \leq i \leq 30$ in \cref{tab:partitions} is contained in the universe of events $\EvtU$. 

Due to the general nature of elements that can be linked to events via the attribute function (in the example, e.g., timestamps and activity labels), we leave the range unspecified, akin to the definition of finite sequence as a function in \cite{Mendelson/2015:IntroductionMathematicalLogic}. We assume the following attributes as mandatory as in the classical process mining literature~\cite{Aalst/2016:ProcessMiningBook:DataScienceinAction}.
%
\begin{assumption}[Mandatory attributes]\label{post:mandatoryattributes}
	The following attributes are total on the universe of events $\EvtU$:
	\begin{inparadesc}
		\item[inter-organizational instance identifier] (or {\CId} for short), a total surjective function $\CIdF : \EvtU \to \CIdU$, where $\CIdU$ is a finite non-empty set we name \emph{instance set};
		\item[activity label\textnormal{,}] a total function $\ActF : \EvtU \to \ActS$, where $\ActS$ is a finite non-empty set we name \emph{activity set};
		\item[timestamp\textnormal{,}] a total function $\TimeF : \EvtU \to \mathbb{N}$, where $\mathbb{N}$ is the set of positive integers.
	\end{inparadesc}
\end{assumption}
%
%Considering the event $\Evt_1$ of our motivating scenario, the {\CIdF} function computed over this event returns the identifier 312, which groups the events constituting Alice's hospitalization in the three organizations. Similarly, the function applications {$\ActF(\Evt_1)$} and the {$\TimeF(\Evt_1)$} link $\Evt_i$ to its activity label (i.e., \texttt{PH}) and its timestamp (i.e., \texttt{2022-07-14T10:36}), respectively.
In the context of our motivating scenario (\cref{tab:partitions}), %when examining the event $\Evt_1$ in \cref{tab:partitions}, the {\CIdF} function yields the inter-organizational identifier 
$\CIdF(\Evt_1) = 312$. This inter-organizational identifier serves the purpose of collating the events associated with Alice's hospitalization across the three organizations. Furthermore, %the function applications 
{$\ActF(\Evt_1)$} and {$\TimeF(\Evt_1)$} map $\Evt_1$ to its activity label (i.e., \texttt{PH}) and timestamp (i.e., \texttt{2022-07-14T10:36}), respectively.
%
\begin{definition}[Event log]\label{def:evt:log}
	Given the universe of events $\EvtU$ as per \cref{def:evt}, let ${\prec \;\subseteq\; \EvtU \times \EvtU}$ be a strict total-order relation defined over $\EvtU$.
	The pair $\EvtL = \left( \EvtU, \prec \right)$ is an \emph{event log}.
\end{definition}
In our example, %we exemplify 
the event log $(\EvtU,\prec)$ is the totally ordered set consisting of all the events $\Evt_1, \ldots, \Evt_{30}$ recorded by the \Actor{Hospital}, the \Actor{Pharmaceutical company}, and the \Actor{Specialized clinic}. We assume that the total order of events can be reconstructed based on the timestamp associated with events, which we express with the following postulate.
%
\begin{assumption}[Timestamp as an order isomorphism]\label{post:order:timestamp}
	Given an event log $\EvtL = \left( \EvtU, \prec \right)$, the timestamp attribute $\TimeF: \EvtU \to \mathbb{N}$ is an \textit{order isomorphism}, i.e., given $\Evt, \Evt' \in \EvtU$ with $\Evt \neq \Evt'$, $\Evt \prec \Evt'$ if and only if $\TimeF(\Evt) < \TimeF(\Evt')$.
\end{assumption}
%
Examining the events $\Evt_{30}$ and $\Evt_{16}$ in our motivating scenario, we can deduce from the order relation established based on their timestamps (i.e., \texttt{2022-07-16T05:06} and \texttt{2022-07-16T07:06}, respectively) that $\Evt_{30}$ precedes $\Evt_{16}$, i.e., ${\Evt_1}\prec{\Evt_{30}}$. This rationale holds true for any generic pair of events selected from the event log.\todo{CHECK ALL TIMESTAMPS IN TABLE 1.}
%

In our setting, events may be recorded by different entities. We name these entities \emph{provisioners} and formally define them as follows.
\begin{definition}[Provisioner]\label{def:provisioner}
	$\LPrvS = \{ \LPrv_1, \ldots, \LPrv_n \}$ is a non-empty finite set of entities (with $n \in \mathbb{N}$) named \emph{provisioner}s. 
    Given the universe of events $\EvtU$ as per \cref{def:evt}, $\PartMap$ is a total surjective function having $\EvtU$ as the domain and $\LPrvS$ as its range. Function $\PartMap$ defines an equivalence relation $\PartEq$ on $\EvtU$ (i.e., $\Evt \PartEq \Evt'$ iff $\Evt \PartMap \LPrv$ and $\Evt' \PartMap \LPrv$) and hence partitions $\EvtU$ into equivalence classes we denote with $\EqCls{\LPrv}$ for every $\LPrv \in \LPrvS$.
\end{definition}
In the depicted scenario, we introduce three provisioners: the \Actor{Hospital}, the \Actor{Pharmaceutical company}, and the \Actor{Specialized clinic}, henceforth referred to as $\LPrv_\textrm{Hsp}$, $\LPrv_\textrm{PhC}$, and $\LPrv_\textrm{SpC}$. Events $\Evt_{1}$ and $\Evt_{19}$ in \cref{tab:trace} fall within the same equivalence class $\EqCls{\LPrv_\textrm{Hsp}}$ as they are both recorded by the \Actor{Hospital}. Consequently, the relation $\Evt_1 \PartEq \Evt_{19}$ holds true. Conversely, $\Evt_{1}$ and $\Evt_{30}$ are recorded by the \Actor{Hospital} and the \Actor{Specialized clinic}, respectively, placing them in distinct equivalence classes, namely $\EqCls{\LPrv_1}$ and $\EqCls{\LPrv_3}$. Therefore, the relation $\Evt_1 \PartEq \Evt_{30}$ does not hold. 
% \todo{Add examples.Done.} 

\begin{definition}[Log partition]\label{def:partition}
	Given an event log $\EvtL = \left( \EvtU, \prec \right)$ as per \cref{def:evt:log}, provisioner $\LPrv \in \LPrvS$ and the equivalence class $\EqCls{\LPrv}$ as per \cref{def:provisioner}, a \emph{log partition} is a pair $\left( \EqCls{\LPrv}, \prec_\LPrv \right)$, where $\prec_\LPrv$ is a total order restricting $\prec$ over the elements in $\EqCls{\LPrv} \subseteq \EvtU$, i.e., $\;\prec_\LPrv$ is $(\prec) \cap \left(\EqCls{\LPrv} \times \EqCls{\LPrv}\right)$.
\end{definition}
%
In our setting, we posit that each organization independently maintains its own log partition. The events  $\Evt_{1}, \Evt_{2}, \ldots, \Evt_{19}$ in the  $\EqCls{\LPrv_{1}}$ equivalence class are contained in the \Actor{Hospital}'s log partition  $\left( \EqCls{\LPrv_{1}}, \prec_{\LPrv_{1}} \right)$. Similarly, the totally ordered sets defined over $\EqCls{\LPrv_{2}} = \left\{\Evt_{20}, \Evt_{21}, \ldots, \Evt_{25} \right\}$ and $\EqCls{\LPrv_{3}} = \left\{ \Evt_{26}, \Evt_{27}, \ldots, \Evt_{30} \right\}$ are the \Actor{Pharmaceutical company} and the \Actor{Specialized clinic}'s log partitions, respectively -- i.e., $\left( \EqCls{\LPrv_{2}}, \prec_{\LPrv_{2}} \right)$ and  $\left( \EqCls{\LPrv_{3}}, \prec_{\LPrv_{3}} \right)$. 

We recall that the restriction of a totally ordered set (like the event log) is a total order. Therefore, the log partition is a totally ordered set, too. hence an event log. In the remainder of this paper, we might name as \textbf{sublog} a restriction of an event log (omitting the reference to the event log whenever clear from the context). For instance, since $\Evt_1 \prec \Evt_2$, then $\Evt_1 \prec_{\LPrv_{1}} \Evt_2$. % The same principle applies likewise to the restrictions $\prec_{\LPrv_{2}}$ and $\prec_{\LPrv_{3}}$ defined onto $\EqCls{\LPrv_{2}}$ and $\EqCls{\LPrv_{3}}$, respectively.
% \todo{Add example}

\begin{definition}[Segment]\label{def:segment}
	Given a log partition $\left( \EqCls{\LPrv}, \prec_\LPrv \right)$ as per \cref{def:partition} and a subset $\{\CId_1, \ldots, \CId_k\} \subseteq \CIdU$ (see \cref{post:mandatoryattributes}) with $1 \leq k \leq |\CIdU|$, $k \in \mathbb{N}$, a \emph{segment} $\Segm = \left( \EqCls{\Segm}, \prec_\Segm \right)$ over $\{\CId_1, \ldots, \CId_k\}$ is a restriction of the log partition to all the events $\Evt_\Segm \in \EqCls{\LPrv}$ such that $\CIdF(\Evt_\Segm) = \CId_i$ for any $1 \leq i \leq k$ (i.e., for every {\CId} in the given subset).
\end{definition}
%
In light of the above, the segment is a totally ordered set, too. Therefore, a segment is an event log (or sublog, depending on the context). Also, notice that a segment can be the whole log partition if the given set of {\CId}s contains all the {\CId}s to which the events in the log partition map. Finally, notice that all events that map to a given $\CId$ in a log partition are in the same segment. In the context of our example, e.g., consider the \Actor{Pharmaceutical company}'s log partition $\left( \EqCls{\LPrv_{2}}, \prec_{\LPrv_{2}} \right)$ and Alice's hospitalization ({\CId} 312). As $\CIdF(\Evt_{20}) = \CIdF(\Evt_{22}) = \CIdF(\Evt_{24}) = 312$ and $ \Evt_{20}, \Evt_{22}, \Evt_{24} \in \EqCls{\LPrv_2} $, all the three aforementioned events reside within the same segment. In other words, any scenario wherein any two events pertaining to Alice's case are allocated to different segments (e.g., $\Evt_{20}$ in \Segm' and $\Evt_{22}$, $\Evt_{24}$ in \Segm'' respectively) is precluded by definition, as per \cref{def:segment}.


\begin{definition}[Merge]\label{def:merge}
	Let
	${\EvtL' = (\EvtU_{\EvtL'}, \prec_{\EvtL'})}$ and
	${\EvtL'' = (\EvtU_{\EvtL''}, \prec_{\EvtL''})}$ be two event logs as per \cref{def:evt:log}. 
	The \emph{merge} of two event logs $\Segm' \Merge \Segm''$ is a pair $\left(\EvtU_{\EvtL'} \cup \EvtU_{\EvtL''}, \prec_{\Merge} \right)$ where $\prec_{\Merge} \supseteq \left( \prec_{\EvtL'} \cup \prec_{\EvtL''} \right)$.
	A merge is \emph{safe} if the two following conditions hold:
	\begin{compactenum}
		\item $\prec_{\Merge}$ is a total order over $\EvtU_{\EvtL'} \cup \EvtU_{\EvtL''}$;
		\item the timestamp attribute $\TimeF: \EvtU_{\EvtL'} \cup \EvtU_{\EvtL''} \to \mathbb{N}$ is an order isomorphism as per \cref{post:order:timestamp}.
	\end{compactenum}
\end{definition}
%
Notice that the merge of two event logs (or sublogs) still is a totally ordered set (and hence an event log, or a sublog) if and only if the merge is safe. In other words, event logs are closed under the safe-merge operation. For instance, a non-safe merge between $\EvtL'$ and $\EvtL''$ could be computed as $\left(\EvtU_{\EvtL'} \cup \EvtU_{\EvtL''}, \prec_{\EvtL'} \cup \prec _{\EvtL''} \right)$. Take, e.g., the segment over {\CId} $312$ in the \Actor{Hospital}'s partition as $\EvtL'$ and the segment over {\CId} $312$ in the \Actor{Pharmaceutical company}'s partition as $\EvtL''$. With a non-safe merge like the one we just mentioned, the order would not be defined for either of $\Evt_{11}$ (in $\EvtL'$) and $\Evt_{20}$ (in $\EvtL''$), among others.

\todo[inline]{The preservation of the total order wrt the original log cannot be guaranteed if we have only two operands, namely the segments. We have to make it a property that only \emph{special} merges have. Surprise surprise, our function will do it by resorting to the timestamping. \\
Notice that we never leave off cases from one of the segments while merging because although the def of safe merge would not impose it, still the merge of two sublogs would be incomplete.}
%
\todo[inline]{Separate the merge op from the iid-selection. The first one MUST be clean, sound, and complete. The second one is an optimisation process. In principle, we would love to ``solve'' the integer max-knapsack problem every time trying to stick in a segment as many iid's as possible given a segsize. However, max-knapsack is NP-hard. Therefore, approximations are welcome. Our approximation is: we take iid's one after the other and fill the bucket up to when segsize is exceeded. The intuition is that we have a $< 1/2$ approximation. Left for future work.}

\begin{comment}
def iid(event):
  bla bla bla
  
caseid = iid
hospitalcaseid = iid
treatmentid = iid

comesichiama = {"hospital" : "caseid", "pharma" : "hospitalcaseid", "clinic" : "treatmentid"}

comesichiama["hospital"](x)
\end{comment}
\begin{comment}
In light of the above, the log partition is a totally ordered set, too. In the example, \ldots\todo{Add example}
Let $\CIdU$ be a finite non-empty set of symbols such that $|\CIdU| \leqslant |\EvtU|$.

, and (blah) as above. The order-preserving union $\Merge: \EvtU^3 \times \EvtU^3 \to \EvtU^3$ of losets is computed as follows: $(\EvtU', \prec') \Merge (\EvtU'', \prec'') = \left(\EvtU' \cup \EvtU'', \prec \cap\, (\EvtU'\cup\EvtU'')^2\right)$.

    

%
We assume that every event be associated with a \emph{case identifier} $\CId \in \CIdU$ via a total surjective function $\CIdF: \EvtU \to \CIdU$ such that the restriction $\prec_\CId \,=\, \prec \cap \{ \Evt \in \EvtU: \CIdF(\Evt) = \CId \}^2$ of total order $\prec$ on all events mapped to the same $\CId$ is strict (i.e., if $\Evt \prec \Evt'$ with $\Evt \neq \Evt'$ and $\CIdF(\Evt) = \CIdF(\Evt')$ then $\Evt' \npreceq \Evt$).
%
In the example, \ldots\todo{Add example}
%
In other words, $\CIdF$ acts as an equivalence relation partitioning $\EvtU$ into $\left\{ \EvtU_\CId \right\}_{\CId \in \CIdU} \subseteq 2^{\EvtU}$ based on the {\CId} to which the events $\Evt \in \EvtU_\CId$ map, and imposing that events are linearly ordered by the restriction of $\prec$ on every $\EvtU_\CId$.
Every pair $\left( \EvtU_\CId, \prec_\CId \right)$ thus represents a finite linearly totally ordered set (or \emph{loset} for brevity) with $\EvtU_\CId \subseteq \EvtU$ and $\prec_\CId \;\subseteq \EvtU_\CId \times \EvtU_\CId \subseteq\; \prec \;\subseteq \EvtU \times \EvtU$.
Let $\left(\EvtU, \prec\right)$ be a loset and $(\EvtU', \prec')$, $(\EvtU'', \prec'')$ two (sub-)losets such that $\EvtU' \cup \EvtU'' \subseteq \EvtU$ and $\EvtU' \cap \EvtU'' = \emptyset$, with $\prec'$ and $\prec''$ being the restrictions of $\prec$ on $\EvtU'$ and $\EvtU''$, respectively:
$\prec' \triangleq \prec \cap \, \EvtU' \times \EvtU'$ and $\prec'' \triangleq \prec \cap \, \EvtU'' \times \EvtU''$.
\begin{definition}[Order-preserving union]\label{def:order-prev:union}
Let $\prec$ be a finite linear order defined over $\EvtU$, and (blah) as above. The order-preserving union $\Merge: \EvtU^3 \times \EvtU^3 \to \EvtU^3$ of losets is computed as follows: $(\EvtU', \prec') \Merge (\EvtU'', \prec'') = \left(\EvtU' \cup \EvtU'', \prec \cap\, (\EvtU'\cup\EvtU'')^2\right)$.
\end{definition}
%
We can thus derive the notion of case $\Case_\CId$ given a $\CId \in \CIdU$ as a loset of events mapping to the same $\CId$ and ordered by the linear restriction $\prec$ of $\prec$ over the events in $\Case_\CId$: $\Case_\CId=\left(\EvtU_\CId, \prec\right)$ where $\Case_\CId = \langle \Evt_1, \ldots, \Evt_{|\Case_\CId|} \rangle$ where $\CIdF(\Evt_i) = \CId \in \CIdU$ for every $i$ s.t.\ $ 1 \leqslant i \leqslant |\Case_\CId|$ and $e_i \prec e_j$ for every $ i \leq j \leqslant |\Case_\CId|$.%
\footnote{We employ the angular-bracket notation here for the sake of simplicity, although it is typically used for sequences. Unlike sequences, cases do not allow for the same event to occur more than once.}
Notice that the cardinality of $\CasU$ and $\CIdU$ coincide.
%
Events are also the domain of a function $\LPrvF: \EvtU \to \LPrvS$ mapping events to log provisioners.
In the example, \ldots\todo{Add example}
We shall denote with $\CasP_\CId$ the loset consisting of every event $\Evt \in \Case_\CId$ such that $\LPrvF(\Evt) = \LPrv$, with the restriction of the strict total order of $\Case_\CId$ on those events.
In the example, \ldots\todo{Add example}
\todo[inline]{Gotta move this one earlier when we introduce the example with the partitioned event log. (Section 4.2??). Clarify the difference between segmentation (given a segsize, i.e., a segment of a case-part in a sublog) and partitioning (of a log into case-parts of sublogs. Then, prove that the pipeline of partitioning and segmentation has its inverse in the union and merge for soundness.} 
%\end{comment}
%
\end{comment}
\end{newj}