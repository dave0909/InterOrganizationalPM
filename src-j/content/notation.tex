\section{Notation}
Given a finite set of events $\EvtU$ and a total-order relation $\preceq$ subset of $\EvtU \times \EvtU$, we identify an event log as the totally ordered set $\left( \EvtU, \preceq \right)$.
In the example, \ldots\todo{Add example}
%
Let $\CIdU$ be a finite non-empty set of symbols such that $|\CIdU| \leqslant |\EvtU|$.
We assume that every event be associated with a \emph{case identifier} $\CId \in \CIdU$ via a total surjective function $\CIdF: \EvtU \to \CIdU$ such that the restriction $\prec_\CId \,=\, \preceq \cap \{ \Evt \in \EvtU: \CIdF(\Evt) = \CId \}^2$ of total order $\preceq$ on all events mapped to the same $\CId$ is strict (i.e., if $\Evt \preceq \Evt'$ with $\Evt \neq \Evt'$ and $\CIdF(\Evt) = \CIdF(\Evt')$ then $\Evt' \npreceq \Evt$).
%
In the example, \ldots\todo{Add example}
%
In other words, $\CIdF$ acts as an equivalence relation partitioning $\EvtU$ into $\left\{ \EvtU_\CId \right\}_{\CId \in \CIdU} \subseteq 2^{\EvtU}$ based on the {\CId} to which the events $\Evt \in \EvtU_\CId$ map, and imposing that events are linearly ordered by the restriction of $\preceq$ on every $\EvtU_\CId$.
Every pair $\left( \EvtU_\CId, \prec_\CId \right)$ thus represents a finite linearly totally ordered set (or \emph{loset} for brevity) with $\EvtU_\CId \subseteq \EvtU$ and $\prec_\CId \;\subseteq \EvtU_\CId \times \EvtU_\CId \subseteq\; \preceq \;\subseteq \EvtU \times \EvtU$.
Let $\left(\EvtU, \prec\right)$ be a loset and $(\EvtU', \prec')$, $(\EvtU'', \prec'')$ two (sub-)losets such that $\EvtU' \cup \EvtU'' \subseteq \EvtU$ and $\EvtU' \cap \EvtU'' = \emptyset$, with $\prec'$ and $\prec''$ being the restrictions of $\prec$ on $\EvtU'$ and $\EvtU''$, respectively.
We define the order-preserving union $\Merge: \EvtU^3 \times \EvtU^3 \to \EvtU^3$ of losets as follows: $(\EvtU', \prec') \Merge (\EvtU'', \prec'') = \left(\EvtU' \cup \EvtU'', \prec \cap\, (\EvtU'\cup\EvtU'')^2\right)$.
\todo{Continue here revising the definition of order preserving union}
%
We can thus derive the notion of case $\Case_\CId$ given a $\CId \in \CIdU$ as a loset of events mapping to the same $\CId$ and ordered by the linear restriction $\prec$ of $\preceq$ over the events in $\Case_\CId$: $\CId=(\EvtU_\CId, \prec)$ where $\Case_\CId = \langle \Evt_1, \ldots, \Evt_{|\Case_\CId|} \rangle$ where $\CIdF(\Evt_i) = \CId \in \CIdU$ for every $i$ s.t.\ $ 1 \leqslant i \leqslant |\Case_\CId|$ and $e_i \prec e_j$ for every $ i \leq j \leqslant |\Case_\CId|$.%
\footnote{We employ the angular-bracket notation here for the sake of simplicity, although it is typically used for sequences. Unlike sequences, cases do not allow for the same event to occur more than once.}
Notice that the cardinality of $\CasU$ and $\CIdU$ coincide.
%
Events are also the domain of a function $\LPrvF: \EvtU \to \LPrvS$ mapping events to log provisioners.
In the example, \ldots\todo{Add example}
We shall denote with $\CasP_\CId$ the loset consisting of every event $\Evt \in \Case_\CId$ such that $\LPrvF(\Evt) = \LPrv$, with the restriction of the strict total order of $\Case_\CId$ on those events.
In the example, \ldots\todo{Add example}
\todo[inline]{Gotta move this one earlier when we introduce the example with the partitioned event log. (Section 4.2??). Clarify the difference between segmentation (given a segsize, i.e., a segment of a case-part in a sublog) and partitioning (of a log into case-parts of sublogs. Then, prove that the pipeline of partitioning and segmentation has its inverse in the union and merge for soundness.} 
%\end{comment}
%
