\section{Introduction}
\label{sec:introduction}
\begin{comment}
\todo[inline]{%
	´CDC: Status corrente secondo me. Abbiamo\ldots
	Un'intro affabile ma che dobbiamo ricalibrare per toglierla dal pantano della TEE e puntare dritto al nostro obiettivo dichiarato e mantenuto, con vista sul potenziale inespresso.\\
	Un motivating scenario che però resta staccato dal resto.\\
	Un related work che però dovrebbe spostarsi sul far capire cosa facciamo di più e di diverso, o di simile ma rivisto.\\
	Una overview eccellente dell'architettura che però non è connessa con l'esempio.\\
	Una overview accurata delle interazioni dei moduli (e quando lo spazio è poco è il caso di fare il merge tra le due cose per risparmiare spazio, come ci dicemmo) -- anche qui senza esempio, dunque capire cosa si fa dove è arduo. Non per me o per te, ma perché conosciamo il lavoro. Chi non lo conosce, non ha assolutamente idea di cosa voglia dire fare il merge degli eventi da tracce provienienti da actor diversi rimettendoli in ordine così da preservare integrità temporale, per esempio.\\
	Una discussion/evaluation ben congegnata in cui però manca la parte quantitativa e la qualitativa è ancora da dettagliare.\\
	Una conclusione.%
}
\end{comment}
In today's business landscape, organizations constantly seek ways to enhance operational efficiency, increase performance, and gain valuable insights to improve their processes. Process mining offers techniques to discover, monitor, and improve business processes by extracting knowledge from chronological records known as \textit{event logs}~\cite{DBLP:books/sp/22/WeerdtW22}.
Process-aware information systems record events referring to activities and interactions within a business process. The vast majority of process mining contributions consider \textit{intra-organizational} settings, in which business processes are executed inside individual organizations. However, organizations increasingly recognize the value of collaboration and synergy in achieving operational excellence. \textit{Inter-organizational} business processes involve several independent organizations cooperating to achieve a shared objective \cite{van2011intra}. Despite the advantages of transparency, performance optimization, and benchmarking that companies can gain, inter-organizational process mining raises challenges that hinder its application. The major issue concerns confidentiality. Companies are reluctant to share private information required to execute process mining algorithms with their partners~\cite{liu2009challenges}. Indeed, letting sensitive operational data traverse organizational boundaries introduces concerns about data privacy, security, and compliance with internal regulations~\cite{muller2021trust}. 
%
%
%1 POSSIBILITA') Allo stato attuale, la letteratura manca di approcci che permettano l'aggregazione di event log da data sources decentralizzate in terze parti computazionali verificabili che garantiscano l'applicazione di algoritmi di process mining in modo thrustworty e secrecy preserving. Per ovviare a questo gap, in questo paper  ....
%2 POSSIBILITÀ) L'adozione di queste tecnologie comporta la definizione di meccanismi dedicati alla provigione sicura degli event log, la loro aggregazione da molteplici sources dentralizzate e la conseguente applicazione di algoritmi all'interno delle TEE in modo trustworty e secrecy preserving. La letteratura attualente manca di tali meccanismi. Per ovviare a questo gap, in questo paper .... 
%3)I protocolli per applicare process mining su data Sources decentralizzate non considerano l'aspetto privacy preserving, che risulta essere uno degli aspetti più invalidanti per le aziende che vogliono beneficiare di tali pratiche. Ti fill this gap...
%
%4)Attualmente, lo stato dell'arte non presenta delle metodologie che consentano di applicare process mining su data Sources decentralizzate in modo trustworthy, preservando la confidenzialità e l'integrità degli event log, una volta che questi varcano i propri confini aziendali. Per questo si ricorre metodologie come SMPC e federated mining che impongono pesanti restrizioni sulla flessibilità degli algoritmi da applicare. To fill this gap ...
To address this issue, the majority of research endeavors have focused thus far on the alteration of input data or of intermediate analysis by-products, with the aim to impede the counterparts from reconstructing the original information sources~\cite{DBLP:journals/dke/FahrenkrogPetersenAW23,DBLP:conf/icpm/Fahrenkrog-Petersen19,elkoumy2020shareprom,elkoumy2020secure}. These preemptive solutions have the remarkable merit of neutralizing information leakage by malicious parties a priori. Nevertheless, they entail an ex-ante information loss, thus compromising downstream process mining capabilities~\cite{DBLP:journals/dke/FahrenkrogPetersenAW23,DBLP:conf/icpm/Fahrenkrog-Petersen19}, or require the execution of computationally heavy protocols undermining scalability~\cite{elkoumy2020shareprom,elkoumy2020secure}.

To overcome these limitations, we propose CONFINE, a novel approach and tool aimed at enhancing collaborative information system architectures with secrecy-preserving process mining capabilities. To secure information secrecy during the exchange and elaboration of data, our solution resorts to \emph{Trusted Execution Environments} (TEEs)~\cite{DBLP:conf/trustcom/SabtAB15}, namely hardware-secured contexts that guarantee code integrity and data confidentiality before, during, and after their utilization. Owing to these characteristics, CONFINE lets information be securely transferred beyond the organizations' borders. Therefore, computing nodes other than the information provisioners can aggregate and elaborate the original, unaltered process data in a secure, externally inaccessible vault. Also, CONFINE is capable of providing these guarantees while demanding scalable computational overhead. 
%
The decentralized architecture of CONFINE supports a four-staged protocol:
%
\begin{inparaenum}[\itshape(i)\upshape]
	\item The initial exchange of preliminary metadata,
	\item the attestation of the mining entities,
	\item the secure transmission and secrecy-preserving merge of encrypted information segments amid multiple parties,
	\item the isolated and verifiable computation of process discovery algorithms on joined data.
\end{inparaenum}
%
We evaluate our proof-of-concept implementation against synthetic and real-world data with a convergence test followed by experiments to assess the scalability of our approach.
Since TEEs operate with dedicated memory pages shielded from access by external entities (operating system included), thus entailing a hardware constraint on computation space, we endow our experiments with an analysis of memory usage, too.





The remainder of this paper is as follows. \Cref{sec:backgroundRelated} provides an overview of related work. % inherent to the theme of inter-organizational process mining. 
In \cref{sec:motivating}, we introduce a motivating use-case scenario in healthcare. We present the CONFINE approach in \cref{sec:design}. We describe the implementation of our approach in \cref{sec:realization}. %, focusing on the employed technologies in our CONFINE protocol and implementation details. 
In \cref{sec:evaluation}, we report on the efficacy and efficiency tests for our solution.
Finally, we conclude our work and outline future research directions in \cref{sec:conclusion}.