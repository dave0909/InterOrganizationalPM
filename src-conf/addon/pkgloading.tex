% For language-specific hyphenations etc.
\usepackage[english]{babel}
% For subfigures
\usepackage{subcaption}
\usepackage{xcolor}
% For nice links
\usepackage{url}
% For playing with colors in tabular environments
\usepackage{colortbl}
% For math symbols, such as \nexists
\usepackage{amssymb}
% For more math symbols, such as \mapsfrom
\usepackage{stmaryrd}
% For advanced graphics
% For equations, arrays of equations, defining operator names, etc.
\usepackage{amsmath}
% For cursive math
\usepackage{mathrsfs}
% For math symbols, such as \nexists
\usepackage{amssymb}
%% For math environments, such as "definition"
%\usepackage{amsthm}
%\theoremstyle{definition}
%\newdefinition{definition}{Definition}[section]
%\newtheorem{theorem}{Theorem}[section]
% For enumerating the line numbers
\usepackage[left]{lineno}
% For diagonally-split fractions (\sfrac)
\usepackage{xfrac}
% For nice (diagonal) fractions
\usepackage{nicefrac}
% For side notes, missing figures and inline to-do's
\usepackage[textsize=scriptsize,backgroundcolor=yellow!40]{todonotes}
% Resize the width of todo-notes on the margins
%\setlength{\marginparwidth}{1.25cm}
% For specifying kewords and acronyms
\usepackage[nonumberlist,acronym,sanitize=none]{glossaries}
\glsdisablehyper
% For commenting out some parts of the text
\usepackage{comment}
% For hyperlinks
\usepackage[pdftex, colorlinks=true, hyperfootnotes=true, hyperindex=true,
            plainpages=false, pagebackref=false, pdfpagelabels=true, pdfstartview=FitH,
            linkcolor=blue, citecolor=blue, urlcolor=blue,
            bookmarks, bookmarksopen, bookmarksdepth=3]{hyperref}
% For smart references
\usepackage[capitalise,nameinlink]{cleveref}
% To have "Figure 3(a)" in place of "Figure 3a" and  "Table 3(a)" in place of "Table 3a"
\captionsetup[subfigure]{subrefformat=simple,labelformat=simple}
    \renewcommand\thesubfigure{(\alph{subfigure})}
\captionsetup[subtable]{subrefformat=simple,labelformat=simple}
    \renewcommand\thesubtable{(\alph{subtable})}
\crefname{algocf}{alg.}{algs.}
\Crefname{algocf}{Algorithm}{Algorithms}
\crefname{section}{Sect.}{Sects.} % Era un poco prolisso
\crefname{Section}{Section}{Sections} % Era un poco prolisso
% TikZ/Pgf advanced graphics
\usepackage{tkz-base}
\usetikzlibrary{decorations.pathmorphing,trees,snakes,arrows,shapes,automata,petri}
% To use inline and other fancy list-like environments (e.g., inparaenum)
\usepackage{paralist}
% To divide a text line into multiple columns
\usepackage{multicol}
% To create good-looking book-style tables
\usepackage{booktabs}
% To play around with list environments
\usepackage{enumitem}
% To create multirow cells in tables
\usepackage{multirow}
% To create rotated cells in tables
\usepackage{rotating}
% To create enumerated lists, whose numbering is reversed
\usepackage{etaremune}
%% To make algorithmic nice-looking pseudocode
\usepackage[ruled,linesnumbered,algo2e]{algorithm2e}
%% For creating side-notes
\usepackage{marginnote}
% For superimposing symbols over one another within math env.
\usepackage{mathtools}
% For strange math symbols like \Dashv
\usepackage{mathabx}
% For adjusting the size of tables to the text width, if necessary: \begin{adjustbox}{max width=\textwidth}
\usepackage{adjustbox}
% For LaTeX if/then statements
\usepackage{ifthen}
% For strike-through cancellations
\usepackage[normalem]{ulem}
%% The lineno packages adds line numbers. Start line numbering with
%% \begin{linenumbers}, end it with \end{linenumbers}. Or switch it on
%% for the whole article with \linenumbers.
\usepackage{lineno}
% For highlighted text
\usepackage{soul}
% To put table environments and co. side by side
\usepackage{floatrow}
\floatsetup[table]{style=plaintop}
\usepackage{listings}
\begin{comment}
	contenuto...
\lstset{ %
%	backgroundcolor=\color{white},   % choose the background color; you must add \usepackage{color} or \usepackage{xcolor}; should come as last argument
	basicstyle=\tiny\ttfamily,       % the size of the fonts that are used for the code
	breakatwhitespace=true,          % sets if automatic breaks should only happen at whitespace
	breaklines=true,                 % sets automatic line breaking
	captionpos=b,                    % sets the caption-position to bottom
	commentstyle=\color{gray},       % comment style
%	escapeinside={\%*}{*)},          % if you want to add LaTeX within your code
	extendedchars=true,              % lets you use non-ASCII characters; for 8-bits encodings only, does not work with UTF-8
	frame=single,	                 % adds a frame around the code
	keepspaces=true,                 % keeps spaces in text, useful for keeping indentation of code (possibly needs columns=flexible)
%	keywordstyle=\color{blue},       % keyword style
%	language=SQL,                    % the language of the code
%	deletekeywords={Time},      % if you want to delete keywords from the given language
%	morekeywords={*,...},            % if you want to add more keywords to the set
	numbers=left,                    % where to put the line-numbers; possible values are (none, left, right)
	numbersep=5pt,                   % how far the line-numbers are from the code
	numberstyle=\tiny\color{gray}, % the style that is used for the line-numbers
%	rulecolor=\color{black},         % if not set, the frame-color may be changed on line-breaks within not-black text (e.g. comments (green here))
	showspaces=false,                % show spaces everywhere adding particular underscores; it overrides 'showstringspaces'
	showstringspaces=false,          % underline spaces within strings only
	showtabs=false,                  % show tabs within strings adding particular underscores
	stepnumber=1,                    % the step between two line-numbers. If it's 1, each line will be numbered
%	stringstyle=\color{mymauve},     % string literal style
	tabsize=2,	                   % sets default tabsize to 2 spaces
%	title=\lstname                   % show the filename of files included with \lstinputlisting; also try caption instead of title
	morecomment=[l]\%
}


\end{comment}

\lstset{
	backgroundcolor=\color{scriptcolor},
	extendedchars=true,
	basicstyle=\fontsize{8pt}{8pt}\selectfont\ttfamily,
	showstringspaces=false,
	showspaces=false,
%	linewidth=19em,
	numbers=left,
	numberstyle=\footnotesize,
	numbersep=5pt,
	tabsize=2,
	breaklines=true,
	showtabs=false,
	captionpos=b,
	%belowskip=-0.2em,
	lineskip = 0.1em
}
\crefname{lstlisting}{Listing}{Listings}
% To add dummy text
\usepackage{lipsum}
% To have newlines in cells, with commands such as \makecell or \thead
\usepackage{makecell}
% For diagonal lines in tables
\usepackage{diagbox}
% For a decent formatting of numbers (and a wonderful system for numeric columns in tables, ``S'')
\usepackage[scientific-notation=false,group-separator={,}]{siunitx}
% To enable text protrusion
\usepackage{microtype}
% For footnote-references: \footref
\usepackage{footmisc}
% For side figures
\usepackage{wrapfig}
%
%\usepackage{xcolor}

\usepackage{listings}
\usepackage[ddmmyyyy]{datetime}
\usepackage{scrextend}
\usepackage{textcase}